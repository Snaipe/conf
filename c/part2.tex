\documentclass[10pt]{beamer}

\usetheme{m}

\usepackage{fontawesome}
\usepackage{setspace}
\usepackage{color}
\usepackage{listings}
\usepackage{booktabs}
\usepackage[scale=2]{ccicons}
\usepackage{textpos}

\setlength{\TPHorizModule}{1cm}
\setlength{\TPVertModule}{1cm}

\definecolor{pgrey}{rgb}{0.46,0.45,0.48}
\definecolor{pblue}{rgb}{0.13,0.13,1}
\definecolor{pgreen}{rgb}{0,0.5,0}

\lstdefinestyle{C++}
{
    language=C++,
    showspaces=false,
    showtabs=false,
    showstringspaces=false,
    basicstyle=\ttfamily,
    keywordstyle=\color{magenta}\ttfamily,
    stringstyle=\color{pgreen}\ttfamily,
    commentstyle=\color{pgrey}\ttfamily,
    morecomment=[l][\color{pgrey}]{\#},
    moredelim=[il][\color{pgrey}]{\#\#\#},
    moredelim=[is][\textcolor{pblue}]{$$}{$$},
    morekeywords={Test,TestSuite,ParameterizedTest,ParameterizedTestParameters}
}

\lstdefinestyle{Rust}
{
    language=C++,
    showspaces=false,
    showtabs=false,
    showstringspaces=false,
    basicstyle=\ttfamily,
    keywordstyle=\color{magenta}\ttfamily,
    stringstyle=\color{pgreen}\ttfamily,
    commentstyle=\color{pgrey}\ttfamily,
    moredelim=[il][\color{pgrey}]{\#\#},
    moredelim=[is][\textcolor{pblue}]{$$}{$$},
    morekeywords={fn}
}

\lstdefinestyle{Java}
{
    language=Java,
    showspaces=false,
    showtabs=false,
    showstringspaces=false,
    basicstyle=\ttfamily,
    keywordstyle=\color{magenta}\ttfamily,
    stringstyle=\color{pgreen}\ttfamily,
    commentstyle=\color{pgrey}\ttfamily,
    morecomment=[l][\color{pgrey}]{\@},
    moredelim=[il][\color{pgrey}]{\#\#},
    moredelim=[is][\textcolor{pblue}]{$$}{$$},
}
\lstset{style=C++}

\usepackage{pgfplots}
\usepgfplotslibrary{dateplot}

\title{Experiments on building high level abstractions with C\newline\newline Part 2: Modern Unit Testing}
\subtitle{A tale of pain and wonder}
\date{4/12/2015}
\author{Franklin "Snaipe" Mathieu -- 2017}
\institute{EPITA -- GCONFS}
%\titlegraphic{\hfill\includegraphics[height=1.5cm]{logo.pdf}}

\begin{document}

\maketitle

\section{The sorry state of C unit testing frameworks}

\plain{Unit testing is awesome!}

\begin{frame}[fragile]{Unit testing is awesome!}

  Java + JUnit:

  \begin{lstlisting}[style=Java]
  @Test
  public void testAThing() {
      Assert.assertTrue(true);
  }
  \end{lstlisting}

  Python + unittest:

  \begin{lstlisting}[language=Python]
  class TestAThing(unittest.TestCase):
    def test(self):
      self.assert(true)
  \end{lstlisting}

  Rust + cargo test:
  \begin{lstlisting}[style=Rust]
  ###[test]
  fn testAThing() {
      assert!(true);
  }
  \end{lstlisting}

\end{frame}

\begin{frame}[fragile]{What about C?}

  Yay, what about C?

  \begin{onlyenv}<1-4>
  \only<2-4>{test1.c:}
  \begin{lstlisting}
  void testAThing(void) {
      assert(1);
  }
  \end{lstlisting}
  \end{onlyenv}

  \begin{onlyenv}<3-4>
  test1.h:
  \begin{lstlisting}
  #ifndef TEST_1_H_
  # define TEST_1_H_
  void testAThing(void);
  #endif // !TEST_1_H_
  \end{lstlisting}
  \end{onlyenv}

  \begin{onlyenv}<4>
  main.c:
  \begin{lstlisting}
  int main(void) {
      testAThing();
      return 0;
  }
  \end{lstlisting}
  \end{onlyenv}

  \begin{onlyenv}<5>
  test1.c:
  \begin{lstlisting}
  void testAThing(void) { /* */ }
  void testSomethingElse(void) { /* */ }
  \end{lstlisting}

  test1.h:
  \begin{lstlisting}
  #ifndef TEST_1_H_
  # define TEST_1_H_
  void testAThing(void);
  void testSomethingElse(void);
  #endif // !TEST_1_H_
  \end{lstlisting}

  main.c:
  \begin{lstlisting}
  int main(void) {
      testAThing();
      testSomethingElse();
      return 0;
  }
  \end{lstlisting}
  \end{onlyenv}

\end{frame}

\plain{Whoops.}

\plain{Let's use a real framework!}

\begin{frame}[fragile]{Let's use CUnit!}

  \begin{lstlisting}
void test(void) { /* */ }

int main(void) {
    if (CUE_SUCCESS != CU_initialize_registry())
        return CU_get_error();

    CU_pSuite s = CU_add_suite("suite", NULL, NULL);
    if (NULL == s
      || (NULL == CU_add_test(s, "test", test)))
        goto cleanup;

    CU_basic_set_mode(CU_BRM_VERBOSE);
    CU_basic_run_tests();

cleanup:
    CU_cleanup_registry();
    return CU_get_error();
}
  \end{lstlisting}

\end{frame}

\plain{How about Check?}

\begin{frame}[fragile]{How about Check?}

  \begin{onlyenv}<1->
  Check and plain C:
  \begin{lstlisting}
  START_TEST (test_name)
  {
    /* unit test code */
  }
  END_TEST
  \end{lstlisting}
  \end{onlyenv}

  \begin{onlyenv}<3->
  Check, m4, and C:
  \begin{lstlisting}
  # suite The Suite
  # tcase The Test Case

  # test  the_test
    const char msg[] = "Hello, world!\n";

    int nc = printf("%s", msg);
    fail_unless(nc == (sizeof msg - 1));
  \end{lstlisting}
  \end{onlyenv}

  \begin{onlyenv}<2>
  \begin{textblock}{3}(-1,1)
    \includegraphics[height=3cm]{wat.png}
  \end{textblock}
  \end{onlyenv}

  \begin{onlyenv}<4->
  \begin{textblock}{3}(-1,-1)
    \includegraphics[height=3cm]{wat.png}
  \end{textblock}
  \end{onlyenv}

  \begin{onlyenv}<5->
  \begin{textblock}{3}(9,-11)
    \includegraphics[angle=120,height=3.5cm]{wat.png}
  \end{textblock}
  \end{onlyenv}

  \begin{onlyenv}<6->
  \begin{textblock}{1}(-1.5,-5)
    \includegraphics[angle=-90,totalheight=1.5cm]{wat.png}
  \end{textblock}
  \end{onlyenv}

\end{frame}

\plain{Let's step back.}

\begin{frame}[fragile]{What do we want?}

  We want:
  \begin{itemize}[<+-|alert@+>]
    \item Automatic test registration
    \item Industrial-grade quality
    \item Let the user in full control
    \item Compatibility on all major platforms (Linux, Windows, Darwin, FreeBSD)
    \item And most importantly: KISS interface, easy and pleasant to use
  \end{itemize}

\end{frame}

\section{Implementing Criterion}

\begin{frame}[fragile]{A short look at executable file formats}

  ELF (Linux), PE (Windows), Mach-O (Darwin):
  \begin{verbatim}
  /¯¯¯¯¯¯¯¯¯¯¯¯¯¯¯¯¯¯¯¯¯¯¯¯¯¯¯¯¯¯¯¯¯¯¯¯¯¯¯¯¯¯¯¯¯¯\
  |                    HEADER                    |
  |______________________________________________|
  |                   SEGMENTS                   |
  |       /¯¯¯¯¯¯¯¯¯¯¯¯¯¯¯¯¯¯¯¯¯¯¯¯¯¯¯¯¯¯¯¯\     |
  |       |            SECTION             |     |
  |       |________________________________|     |
  |       |                                |     |
  |       |            SECTION             |     |
  |       |________________________________|     |
  |       |                                |     |
  |       |            SECTION             |     |
  |       \________________________________/     |
  |                                              |
  \______________________________________________/
  \end{verbatim}
\end{frame}

\begin{frame}[fragile]{Sections}

  \begin{itemize}[<+-|alert@+>]
    \item Sections may contain arbitrary data.
    \item Nonstandard sections are allowed and may hold user data.
    \item The begining and end of a section are known at link time,
          and run time if the binary is not stripped.
  \end{itemize}

  \begin{onlyenv}<4->
  We can use this to automatically discover our tests!

  Put our test data in a section, then iterate over the section with pointer
  arithmetic.
  \end{onlyenv}

\end{frame}

\begin{frame}[fragile]{Doing things manually}

  User code:
  \begin{lstlisting}
struct test {
  const char *name;
  void (*fn)(void);
};

static void my_test_fn(void) { /* */ }

const static struct test my_test_data = {
        .name "my_test",
        .fn = my_test_fn
    };

__attribute__((section("tests")))
const static struct test *my_test_ptr = &my_test_data;

  \end{lstlisting}

\end{frame}

\begin{frame}[fragile]{Doing things manually}

  \begin{lstlisting}
#define Test(Name)                              \
###    static void Name ## _fn(void);              \
###    const static struct test Name ## _data = {  \
###            .name #Name,                        \
###            .fn = Name ## _fn                   \
###        };                                      \
###    __attribute__((section("tests")))           \
###    const static struct test *Name ## _ptr      \
###        = &Name ## _data;                       \
###    static void Name ## _fn(void)
  \end{lstlisting}

  \begin{onlyenv}<2->
  \begin{lstlisting}
  Test(my_test) { /* */ }
  \end{lstlisting}
  \end{onlyenv}
\end{frame}

\begin{frame}[fragile]{Tackling the missing main}

  Library code:
  \begin{lstlisting}
  static struct test *__start_tests;
  static struct test *__stop_tests;

  int main(void) {
      for (struct test **t = &__start_tests;
           t < &__stop_tests;
           ++t) {

          if (*t == NULL)
              continue;

          printf("Running test '%s'.\n'", (*t)->name);
          (*t)->fn();
      }
      return 0;
  }
  \end{lstlisting}

\end{frame}

\begin{frame}[fragile]{Tackling the missing main}

  Where do we put the main? \pause{}

  We could define a \verb|CRITERION_DEFINE_MAIN| macro, but this is suboptimal.

  Instead, let's remember that \verb|main| is just a symbol, and should get
  resolved by the linker wherever it is.

  \pause{}

  We can then put our main in a separate library, and this ``default'' main will
  be picked up if the user does not define their own main.

\end{frame}

\begin{frame}[fragile]{Looking back at what we did}

  User code:
  \begin{lstlisting}
  #include <criterion/criterion.h>

  Test(foo) {}

  Test(bar) {
      assert(0);
  }
  \end{lstlisting}

  Usage:
  \begin{verbatim}
  $ cc -o test test.c -lcriterion
  $ ./test
  Running test 'foo'.
  Running test 'bar'.
  test: test.c:6: main: Assertion `0' failed.
  \end{verbatim}
\end{frame}

\plain{Wow.}

\section{Enhancing the library}

\begin{frame}[fragile]{Sandboxing for the greater good}
  A unit test should be self-contained and isolated from other parts of the system.

  For the reporting to be relevant, I expect my test runner to be able to complete
  gracefully all the tests, even if they crash.

  \pause{}

  In other words, this shouldn't prevent the other tests from running:
  \begin{lstlisting}
  Test(crash) {
      *((int *) NULL) = 42;
  }
  \end{lstlisting}

  \pause{}

  We need to sandbox each test.

  But how could we recover from errors such as a segmentation fault?

  \pause{}

  Short answer: you can't. After a sigsegv hits, the best thing you can do is exit.

\end{frame}

\begin{frame}[fragile]{Forking around at the speed of sound}

  Let's isolate the test into its own forked process!

  \begin{itemize}
    \item The test runs in its own address space
    \item File handles/descriptors are bound to the process
    \item Threads are local to the process
    \item All of the above points are duplicated from the parent process
    \item For parent/child communication we use a pipe
  \end{itemize}

  With this, even in the event of major corruption or the end of the universe,
  we are guaranteed that the test runner will always be up and running until
  the very end.

  \pause{}

  Problem: there is no \verb|fork()| or \verb|pipe(int[2])| on Windows...

  We would like to avoid using Cygwin.

\end{frame}

\begin{frame}[fragile]{Woe32 strikes yet again}

  Welcome to hell.

  How Cygwin implements fork() in a nutshell:
  \begin{itemize}
    \item Create a suspended process
    \item Copy all memory maps from the parent to the child
    \item Copy processor context, set the IP register
    \item Resume child and return
  \end{itemize}

  \pause{}

  However, we can't do this, because we do not control any of the initialization
  normally provided by the cygwin toolchain.

  It's also horrible to debug.

\end{frame}

\begin{frame}[fragile]{Woe32 strikes yet again}

  What I ended up doing:

  \begin{itemize}
    \item Create a suspended process
    \item Copy a custom heap from the parent to the child
    \item Copy test data into a named mapping
    \item Resume child, child opens the mapping and initialize itself.
    \item In parent, wait for the child to be initialized, then return its PID
  \end{itemize}

  \pause{}

  This is easier to maintain and debug -- however, we can't no longer
  universally guarantee to have the test inheriting its parent's address space.

  \pause{}

  This is why we introduce a custom heap, and \verb|cr_malloc|/\verb|cr_free|
  to manipulate this heap on Windows, or simply call \verb|malloc|/\verb|free|
  on other systems.

\end{frame}

\begin{frame}[fragile]{Enhancing the API}

  \begin{onlyenv}<1>
  \begin{lstlisting}
  Test(bar) {}

  Test(foo) {
      assert(0);
  }
  \end{lstlisting}
  \end{onlyenv}

  \begin{onlyenv}<2>
  \begin{lstlisting}
  Test(suite, bar) {}

  Test(suite, foo) {
      assert(0);
  }
  \end{lstlisting}
  \end{onlyenv}

  \begin{onlyenv}<3>
  \begin{lstlisting}
  Test(suite, crash, .signal = SIGSEGV) {
      *((int *) NULL) = 42;
  }

  Test(suite, foo, .description = "Testing the foos") {
      assert(0);
  }
  \end{lstlisting}
  \end{onlyenv}

  \begin{onlyenv}<4>
  \begin{lstlisting}
  void setup(void) { /* */ }
  void teardown(void) { /* */ }

  Test(suite, crash,
        .signal = SIGSEGV,
        .init = setup,
        .fini = teardown) {
      *((int *) NULL) = 42;
  }

  Test(suite, foo,
        .description = "Testing the foos",
        .init = setup,
        .fini = teardown) {
      assert(0);
  }
  \end{lstlisting}
  \end{onlyenv}

  \begin{onlyenv}<5>
  \begin{lstlisting}
  void setup(void) { /* */ }
  void teardown(void) { /* */ }

  TestSuite(suite, .init = setup, .fini = teardown);

  Test(suite, crash, .signal = SIGSEGV) {
      *((int *) NULL) = 42;
  }

  Test(suite, foo, .description = "Testing the foos") {
      assert(0);
  }
  \end{lstlisting}
  \end{onlyenv}

  \begin{onlyenv}<6>
  \begin{lstlisting}
  void setup(void) { /* */ }
  void teardown(void) { /* */ }

  TestSuite(suite, .init = setup, .fini = teardown);

  Test(suite, crash, .signal = SIGSEGV) {
      *((int *) NULL) = 42;
  }

  Test(suite, foo, .description = "Testing the foos") {
      cr_assert(1);
      cr_assert(0, "Assertion message");
  }
  \end{lstlisting}
  \end{onlyenv}

  \begin{onlyenv}<7>
  \begin{lstlisting}
  void setup(void) { /* */ }
  void teardown(void) { /* */ }

  TestSuite(suite, .init = setup, .fini = teardown);

  Test(suite, crash, .signal = SIGSEGV) {
      *((int *) NULL) = 42;
  }

  Test(suite, foo, .description = "Testing the foos") {
      cr_assert(1);
      cr_expect_eq(1, 0); // fails but does not abort
      cr_assert(0, "A %s format string", "cool");
  }
  \end{lstlisting}
  \end{onlyenv}

\end{frame}

\section{Providing sane defaults and reporting tools}

\begin{frame}[fragile]{CLI and friends}

  Since we control the main, we can provide
  sane tooling that every programmer should expect of
  their unit test runner:

  \begin{itemize}
    \item Control the verbosity level (\verb|--verbose[=N]|)
    \item Filtering the running tests with pattern matching (\verb|--pattern PATTERN|)
    \item List the tests (\verb|--list|)
    \item Fail fast: if one test fails, don't run the others (\verb|--fail-fast|)
    \item And more...
  \end{itemize}

\end{frame}

\begin{frame}[fragile]{Reporting formats}

  In addition to the normal CLI report, we can add multiple test report formats:

  \begin{itemize}
    \item TAP -- Test Anything Protocol (Compatible with a lot of test drivers)
    \item JUnit XML (Compatible with CI services like Jenkins)
    \item Json (For custom webservices)
  \end{itemize}

\end{frame}

\section{Raising the bar on the testing tools}

\begin{frame}[fragile]{Multiple language compatibility}

  Tests in \only<1>{C}\only<2>{C++}\only<3>{Objective-C}\only<4->{*}:

  \begin{lstlisting}
  #include <criterion/criterion.h>

  Test(sample, simple) {
      cr_assert(0, "Hello, World.");
  }
  \end{lstlisting}

  \begin{onlyenv}<4>
  The interface is unified, but there are extensions for C++ (assertions on
  exceptions, catching unhandled exceptions, ...)
  \end{onlyenv}

\end{frame}

\begin{frame}[fragile]{Parameterized tests}

  Parameterized tests are tests that take a parameter from a finite dataset.

  \begin{lstlisting}
  ParameterizedTestParameters(suite, test) {
      static int arr[] = { 1, 2, 3 };
      $$size_t$$ len = sizeof (arr) / sizeof (int);
      return cr_make_param_array(int, arr, len);
  }

  ParameterizedTest(int *param, suite, test) {
      cr_assert_lt(*param, 4);
  }
  \end{lstlisting}

  \pause{}

  Useful for testing and validating the \textbf{same logic} over a \textbf{finite set of data}.

\end{frame}

\begin{frame}[fragile]{Theories}

  Theories are another kind of test, designed for validating a result from
  a set of axioms.

  \pause{}

  Useful for testing properties that makes sense for \textbf{any kind of data}.

  \pause{}

  Vanilla unit testing:

  \begin{lstlisting}
  Test(algebra, mul) {
    cr_assert_eq(div(mul(2, 3), 3), 2);
  }
  \end{lstlisting}

  But this test does not really make sense!

  Should test for any kind of values, not just 2 and 3.

\end{frame}

\begin{frame}[fragile]{Theories}

  With a theory:

  \begin{lstlisting}
  Theory((int lhs, int rhs), algebra, mul) {
      cr_assume_neq(rhs, 0);
      cr_assert_eq(div(mul(lhs, rhs) rhs), lhs);
  }
  \end{lstlisting}

  \pause{}

  When writing the test, we don't really care about \verb|lhs| or \verb|rhs|

  \pause{}

  However, we \textbf{assume} that \verb|rhs| is nonzero.

  \pause{}

  We pass interesting values to \verb|lhs| and \verb|rhs|.

  \begin{lstlisting}
  TheoryDatapoints(algebra, mul) = {
      DataPoints(int, 0, -1, 1, INT_MAX, INT_MIN),
      DataPoints(int, 0, -1, 1, INT_MAX, INT_MIN),
  };
  \end{lstlisting}

\end{frame}

\plain{What's next?}

\begin{frame}[fragile]{What's next?}
  \begin{itemize}
    \item Better float assertions (strategy choice between ULPs, Absolute epsilons, ...)
    \item Concolic testing
    \item Testing for embedded programming
  \end{itemize}

  \begin{center}
    \includegraphics[height=50pt]{criterion.png}

    \small (Logo designed by @pbouigue on twitter) \normalsize

    Any sensible contribution is welcome.

    \large \faGithub \normalsize\hspace{0.7em}
    \url{https://snai.pe/git/criterion}

    \large \faEnvelope \normalsize\hspace{0.7em}
    \verb|franklinmathieu+criterion@gmail.com|

    \verb|Snaipe| on \verb|#criterion| or \verb|#criterion-dev| @ \url{irc.freenode.net}

  \end{center}
\end{frame}

\plain{Questions?}

\end{document}
