\documentclass[10pt]{beamer}

\usetheme{m}

\let\Tiny\tiny

\usepackage{fontawesome}
\usepackage{setspace}
\usepackage{color}
\usepackage{listings}
\usepackage{booktabs}
\usepackage[scale=2]{ccicons}
\usepackage{textpos}
\usepackage{fancyvrb}

\newcommand{\vrbalert}[2][]{%
  \if\relax\detokenize{#1}\relax%
    \alert{#2}%
  \else
    \alert<#1>{#2}%
  \fi}

\setlength{\TPHorizModule}{1cm}
\setlength{\TPVertModule}{1cm}

\definecolor{pgrey}{rgb}{0.46,0.45,0.48}
\definecolor{pblue}{rgb}{0.13,0.13,1}
\definecolor{pgreen}{rgb}{0,0.5,0}

\lstdefinestyle{C++}
{
    language=C++,
    showspaces=false,
    showtabs=false,
    showstringspaces=false,
    basicstyle=\ttfamily,
    keywordstyle=\color{magenta}\ttfamily,
    stringstyle=\color{pgreen}\ttfamily,
    commentstyle=\color{pgrey}\ttfamily,
    morecomment=[l][\color{pgrey}]{\#},
    moredelim=[il][\color{pgrey}]{\#\#\#},
    moredelim=[is][\textcolor{pblue}]{$$}{$$},
}

\lstset{style=C++}

\usepackage{pgfplots}
\usepgfplotslibrary{dateplot}

\title{Experiments on building high level abstractions with C\newline\newline Part 3: Reflection}
\subtitle{Mirror mirror on the wall, am I not the prettiest of them all?}
\date{4/12/2015}
\author{Franklin "Snaipe" Mathieu -- 2017}
\institute{EPITA -- GCONFS}
%\titlegraphic{\hfill\includegraphics[height=1.5cm]{logo.pdf}}

\begin{document}

\maketitle

\section{What is reflection?}

\begin{frame}[fragile]{Definitions}
  $$\text{Reflection} = \only<1>{\text{\_\_\_\_\_\_\_\_\_\_\_\_}}\only<2->{\text{\alert<2>{Introspection}}} + \only<1-2>{\text{\_\_\_\_\_\_\_\_\_\_\_}}\only<3->{\text{\alert<3>{Intercession}}}$$

  \only<2->{\alert<2>{Introspection}: The ability of a program to examine itself} ~

  \only<3->{\alert<3>{Intercession}: The ability of a program to change its state and meaning} ~
  \only<1-2>{\newline}

  \only<4>{Known implementors are Java, C\#, Python, Ruby, PHP, ...} ~
\end{frame}

\begin{frame}[fragile]{Reflection}

  To achieve reflection, we need sufficient information on various actors:

  \begin{itemize}
    \item Types
    \item Functions
    \item Variables
  \end{itemize}

  \pause{}

  Thankfully, debuggers also need these informations to be useful.

\end{frame}

\section{Implementing Insight}

\begin{frame}[fragile]{DWARF}

  DWARF is a standardized debugging format, designed along with ELF.

  Information is stored (and described) as a tree, but contains backreferences
  on nodes, making it effectively more like a graph.

  \pause{}

  libdwarf can be used to easily parse and iterate over the tree.

\end{frame}

\begin{frame}[fragile]{Anatomy of a simple C program}

  test.c:
  \begin{lstlisting}
  struct T { int field; double other_field; };
  static struct T var;

  int main(void) {}
  \end{lstlisting}

  \pause{}

  DWARF information:
  \begin{Verbatim}[commandchars=\\\{\},fontsize=\small,frame=single]
  DW_TAG_compile_unit
    DW_AT_producer  GNU C11 5.2.0 -mtune=generic 
                      -march=x86-64 -g
    DW_AT_language  DW_LANG_C99
    DW_AT_name      test.c
    DW_AT_comp_dir  /home/snaipe
    DW_AT_low_pc    0x004004b6
    DW_AT_high_pc   <offset-from-lowpc>11
    DW_AT_stmt_list 0x00000000
  \end{Verbatim}
\end{frame}

\begin{frame}[fragile]{Anatomy of a simple C program}
\begin{Verbatim}[commandchars=\\\{\},fontsize=\small,frame=single]
<0x2d> DW_TAG_structure_type
         DW_AT_name: \vrbalert[2]{"T"}
         DW_AT_byte_size: \vrbalert[3]{0x10}
         DW_AT_decl_file: \vrbalert[4]{0x1}
         DW_AT_decl_line: \vrbalert[4]{0x1}
<0x37>   DW_TAG_member
           DW_AT_name: \vrbalert[2]{"field"}
           DW_AT_decl_file: \vrbalert[4]{0x1}
           DW_AT_decl_line: \vrbalert[4]{0x1}
           DW_AT_type: \vrbalert[6]{<0x50>}
           DW_AT_data_member_location: \vrbalert[5]{0}
<0x43>   DW_TAG_member
           DW_AT_name: \vrbalert[2]{"other_field"}
           DW_AT_decl_file: \vrbalert[4]{0x1}
           DW_AT_decl_line: \vrbalert[4]{0x1}
           DW_AT_type: \vrbalert[6]{<0x57>}
           DW_AT_data_member_location: \vrbalert[5]{8}
\end{Verbatim}
\end{frame}

\begin{frame}[fragile]{Anatomy of a simple C program}
\begin{Verbatim}[commandchars=\\\{\},fontsize=\small,frame=single]
\vrbalert{<0x50>} DW_TAG_base_type
         DW_AT_byte_size: 0x4
         DW_AT_encoding: DW_ATE_signed
         DW_AT_name: "int"
\vrbalert{<0x57>} DW_TAG_base_type
         DW_AT_byte_size: 0x8
         DW_AT_encoding: DW_ATE_float
         DW_AT_name: "double"
\end{Verbatim}
\end{frame}

\begin{frame}[fragile]{Anatomy of a simple C program}
\begin{Verbatim}[commandchars=\\\{\},fontsize=\small,frame=single]
<0x5e> DW_TAG_subprogram
         DW_AT_external: yes(1)
         DW_AT_name: "main"
         DW_AT_decl_file: 0x1
         DW_AT_decl_line: 0x4
         DW_AT_prototyped: yes(1)
         DW_AT_type: <0x50>
         DW_AT_low_pc: \vrbalert[2]{0x004004b6}
         DW_AT_high_pc: \vrbalert[2]{<offset-from-lowpc>11}
         DW_AT_frame_base: \vrbalert[3]{len 0x1: DW_OP_call_frame_cfa}
<0x7b> DW_TAG_variable
         DW_AT_name: "var"
         DW_AT_decl_file: 0x1
         DW_AT_decl_line: 0x2
         DW_AT_type: <0x2d>
         DW_AT_location: \vrbalert[3]{len 0x9: DW_OP_addr 0x006008c0}
\end{Verbatim}
\end{frame}

\begin{frame}[fragile]{Implementation}

  We need to process this tree and turn it into usable metadata for us.

  \pause{}

  \begin{lstlisting}
typedef struct insight_type_s   *$$insight_type_info$$;
typedef struct insight_field_s  *$$insight_field_info$$;
  \end{lstlisting}

  \pause{}

  \begin{lstlisting}
$$insight_type_info$$ insight_typeof_str(const char *);
$$insight_type_info$$ insight_typeof_addr(void *);
  \end{lstlisting}
  \pause{}
  \begin{lstlisting}
const char *insight_type_name($$insight_type_info$$);
$$size_t$$ insight_type_size($$insight_type_info$$);
  \end{lstlisting}
  \pause{}
  \begin{lstlisting}
$$insight_field_info$$ insight_field($$insight_type_info$$,
                                 const char *);
void insight_field_set($$insight_field_info$$,
                       void *instance, void *data,
                       size_t size);
void insight_field_get($$insight_field_info$$,
                       void *instance, const void *data,
                       size_t size);
  \end{lstlisting}

\end{frame}

\begin{frame}[fragile]{Introducing type\_of}

  we need a practical way of getting the type of anything.

  \pause{}

  GNU C \verb|__typeof__| is the perfect candidate.

  \begin{onlyenv}<2>
  \small\begin{lstlisting}
#define type_of(Thing) ({ \
###    static __typeof__(Thing) *insight_typeof_dummy = NULL; \
###    insight_typeof_addr(&insight_typeof_dummy); \
###  })
  \end{lstlisting}\normalsize
  \end{onlyenv}

  \begin{onlyenv}<3>
  \small\begin{lstlisting}
#ifdef __GNUC__
# define type_of(Thing) ({ \
###    static __typeof__(Thing) *insight_typeof_dummy = NULL; \
###    insight_typeof_addr(&insight_typeof_dummy); \
###  })
#else
# define type_of(Thing) (insight_typeof_str(#Thing))
#endif
  \end{lstlisting}\normalsize
  \end{onlyenv}

\end{frame}

\begin{frame}[fragile]{Some short examples}
  \begin{lstlisting}
  insight_type_info void_t = type_of(void);
  insight_type_info int_t = type_of(int);
  insight_type_info struct_t = type_of(struct foo);
  \end{lstlisting}

  \pause{}

  \begin{lstlisting}
  insight_type_info int_t = type_of(1);
  insight_type_info long_t = type_of(1L);
  insight_type_info double_t = type_of(3.14);
  insight_type_info char_arr_t = type_of("Foo");
  \end{lstlisting}
\end{frame}

\begin{frame}[fragile]{Some short examples}

  \begin{lstlisting}
struct T {
  int i;
};

int main(void) {
  struct test_struct t = {24};

  insight_type_info type = type_of(struct T);
  insight_field_info field = insight_field(type, "i");

  printf("t.i = %d\n", t.i);
  int i = 42;
  insight_field_set(field, &t, &i, sizeof (i));
  printf("t.i = %d\n", t.i);

  return 0;
}
  \end{lstlisting}
\end{frame}

\plain{What about C++?}

\begin{frame}[fragile]{On the case of C++}

  We can build a similar API for C++:

  \begin{lstlisting}
class StructInfo : virtual public TypeInfo,
                   virtual public Container {
public:
  virtual Range<MethodInfo> methods() const = 0;
  virtual MethodInfo& method(std::string) const = 0;
  virtual Range<FieldInfo> fields() const = 0;
  virtual FieldInfo& field(std::string name) const = 0;
  virtual WeakRange<StructInfo> supertypes() const = 0;
  virtual StructInfo& supertype(std::string) const = 0;
  virtual bool is_supertype(const TypeInfo&) const = 0;
  virtual bool is_ancestor(const TypeInfo&) const = 0;
};
  \end{lstlisting}

\end{frame}

\begin{frame}[fragile]{Some C++ examples}

  \begin{lstlisting}
class TestClass {
public:
    TestClass(int f);
    int get_field();
private:
    int field;
};
  \end{lstlisting}

  \pause{}

  \begin{lstlisting}
Insight::StructInfo& type = type_of(TestClass);
  \end{lstlisting}

  \pause{}

  \begin{lstlisting}
int& field = type.field("field").get<int>(instance);
field = 55;
  \end{lstlisting}

  \begin{lstlisting}
type.method("get_field").call<int>(instance)
  \end{lstlisting}

\end{frame}

\begin{frame}[fragile]{Using reflection for foreign interfaces}

  We could:

  \begin{itemize}[<+-|alert@+>]
    \item load an external library and automatically discover what's in it.
    \item Use functions and types without their definitions
    \item Call functions in language A from language B
    \item Manipulate concepts foreign to the current language (e.g. C++ classes
          in C code)
    \item Monkey-patch our types and vtables for profit
  \end{itemize}

  \pause{}

  These are not (yet) implemented, but are entierly possible.

\end{frame}

\section{Implementing annotations}

\begin{frame}[fragile]{Annotations for class metadata}

  \begin{onlyenv}<1>
  Java:
  \begin{lstlisting}[language=Java]
  @Entity
  @Table(name = "events")
  public class Event {
      @Id
      private long id;

      private String name;
  }
  \end{lstlisting}
  \end{onlyenv}

  \begin{onlyenv}<2>
  C++:
  \begin{lstlisting}
  $(Entity)
  $(Table, .name = "events")
  struct event {
      $(Id)
      size_t id;

      std::string name;
  };
  \end{lstlisting}
  \end{onlyenv}

\end{frame}

\plain{O\_o}

\begin{frame}[fragile]{Annotations for class metadata}

  \verb|$| is a nonstandard albeit valid identifier on the GNU compiler family.

  \pause{}

  We define the following:

  \begin{lstlisting}
  #define $(Name, ...)                        \
  ###  static const struct Name <unique_id> = {  \
  ###      __VA_ARGS__                           \
  ###  };
  \end{lstlisting}

  \pause{}

  The static constant holds the actual metadata.

  \pause{}

  The constant is elected as an annotation when traversing the DWARF tree.

  It affects the entry directly below it.

  \pause{}

  For portability purposes, we may choose to conditionally define \verb|$| and
  use a more standard identifier.
\end{frame}

\plain{What's next?}

\begin{frame}[fragile]{What's next?}

  \begin{center}
    \large Get out of alpha and release v1.0! \normalsize

    \includegraphics[height=50pt]{insight.png}

    \small (Logo designed by @pbouigue on twitter) \normalsize

    Any sensible contribution is welcome.

    \large \faGithub \normalsize\hspace{0.7em}
    \url{https://snai.pe/git/insight}

    \large \faEnvelope \normalsize\hspace{0.7em}
    \verb|franklinmathieu+insight@gmail.com|

    \verb|Snaipe| on \verb|#criterion-dev| @ \url{irc.freenode.net}

  \end{center}
\end{frame}

\plain{Questions?}

\end{document}
