\documentclass[10pt]{beamer}

\usetheme{m}

\usepackage{fontawesome}
\usepackage{setspace}
\usepackage{color}
\usepackage{listings}
\usepackage{booktabs}
\usepackage[scale=2]{ccicons}

\definecolor{pgrey}{rgb}{0.46,0.45,0.48}
\definecolor{pblue}{rgb}{0.13,0.13,1}
\definecolor{pgreen}{rgb}{0,0.5,0}

\lstdefinestyle{C++}
{
    language=C++,
    showspaces=false,
    showtabs=false,
    showstringspaces=false,
    basicstyle=\ttfamily,
    keywordstyle=\color{magenta}\ttfamily,
    stringstyle=\color{pgreen}\ttfamily,
    commentstyle=\color{pgrey}\ttfamily,
    morecomment=[l][\color{pgrey}]{\#},
    moredelim=[il][\color{pgrey}]{\#\#},
    moredelim=[is][\textcolor{pblue}]{$$}{$$},
}
\lstset{style=C++}

\usepackage{pgfplots}
\usepgfplotslibrary{dateplot}

\title{Experiments on building high level abstractions with C\newline\newline Part 1: Smart Pointers}
\subtitle{A tale of pain and wonder}
\date{4/12/2015}
\author{Franklin "Snaipe" Mathieu -- 2017}
\institute{EPITA -- GCONFS}
%\titlegraphic{\hfill\includegraphics[height=1.5cm]{logo.pdf}}

\begin{document}

\maketitle

\section{Smart pointers for GNU C99}

\begin{frame}{Smart pointers for GNU C99}

  Memory management is an unsolved problem.

  Known techniques rely on RAII, Reference counting, Garbage collection

  There is no silver bullet, each has its drawbacks

\end{frame}

\begin{frame}[fragile]{A case study on C++}

  C++ solves the problem with RAII and smart pointers:

  \pause{}

  File I/O and RAII:

  \begin{lstlisting}
  {
    $$std::fstream$$ file("foo.txt");
    file << "Hello world";
  } // file is closed when exiting the scope
  \end{lstlisting}

  \pause{}

  Memory allocation:

  \begin{lstlisting}
  {
    auto ptr = std::make_unique<int>(42);
    std::cout << *ptr;
  } // the memory is freed when exiting the scope
  \end{lstlisting}

\end{frame}

\begin{frame}[fragile]{RAII with C}

  There are no RAII facilities in standard C -- cleanup is mainly done with goto

  \begin{lstlisting}
  $$FILE$$ *f = fopen("foo.txt", "w");
  if (!f)
    goto cleanup;

  // do things

  if (error_case)
    goto cleanup;

  // do more things

cleanup:
  if (f) fclose(f);
  \end{lstlisting}

\end{frame}

\begin{frame}[fragile]{Extensions to the rescue}

  Thankfully, some compilers leverage these limitations with extensions:

  \pause{}

  MSVC: Structured exception handling and try/finally

  \begin{lstlisting}
  $$FILE$$ *f = NULL;
  __try {
      f = fopen("foo.txt", "w");
  } __finally {
      if (f) fclose(f);
  }
  \end{lstlisting}

  \pause{}

  GCC, Clang/LLVM, ICC: cleanup attribute

  \begin{lstlisting}
  void fclose_stack($$FILE$$ **f) { if (*f) fclose(*f); }

  __attribute__((cleanup(fclose_stack)))
  $$FILE$$ *f = fopen("foo.txt", "w");
  \end{lstlisting}
\end{frame}

\begin{frame}[fragile]{Building an automatic pointer}

  The cleanup attribute provides the basic building block for an automatic pointer:

  \begin{lstlisting}
  void free_stack($$void$$ **f) {
      free(*f);
  }

  #define auto_ptr __attribute__((cleanup(free_stack)))

  auto_ptr int *val = malloc(sizeof (int));
  *val = 31>174;
  \end{lstlisting}

\end{frame}

\plain{But it's not good enough.}

\begin{frame}[fragile]{A few issues}

  \begin{lstlisting}
  auto_ptr int *val = malloc(sizeof (int));
  *val = 24;
  \end{lstlisting}

  VS.

  \begin{lstlisting}
  auto ptr = std::make_unique<int>(24);
  \end{lstlisting}

  \begin{itemize}
    \item Not convenient to use
    \item No support for structure cleanup
    \item Resource not shareable
  \end{itemize}

  In other words, it's useless garbage.

\end{frame}

\begin{frame}[fragile]{Implementing destructors}

  We need to store a destructor function.

  We need to store metadata.

  We need a custom allocator.

  Introducing smalloc/sfree:

  \begin{lstlisting}
  void *smalloc($$size_t$$ size, void (*dtor)(void*));
  void sfree(void *);
  #define smart __attribute__((cleanup(sfree_stack)))
  \end{lstlisting}

  \begin{verbatim}
|¯¯¯¯¯¯¯¯¯¯|¯¯¯¯¯¯¯¯¯|¯¯¯¯¯¯¯¯¯¯¯¯¯¯¯¯¯¯¯¯¯¯¯//¯¯¯¯¯¯|
| metadata | padding | actual data          //       |
|__________|_________|_____________________//________|
^                    ^ returned address (word aligned)
|_ start of allocated
   block
  \end{verbatim}

\end{frame}

\begin{frame}[fragile]{Simple implementation}
  \begin{lstlisting}
  typedef void ($$f_dtor$$)(void*);
  typedef struct {
      $$f_dtor$$ *dtor;
  } $$s_meta$$;

  void *smalloc($$size_t$$ size, $$f_dtor$$ *dtor) {
      $$size_t$$ prefix_sz = align(sizeof (s_meta));
      $$s_meta$$ *ptr = malloc(size + prefix_sz);
      ptr->dtor = dtor;
      return (char *) ptr + prefix_sz;
  }

  void sfree(void *ptr) {
      $$s_meta$$ *meta = get_meta(ptr);
      if (meta->dtor) meta->dtor(ptr);
      free(ptr);
  }
  \end{lstlisting}
\end{frame}

\begin{frame}[fragile]{Enhancing smalloc}

  \begin{onlyenv}<1>
    \begin{lstlisting}
  void *smalloc($$size_t$$ size, void (*dtor)(void*));
    \end{lstlisting}
  \end{onlyenv}

  \begin{onlyenv}<2-3>
    \begin{lstlisting}
  void *smalloc($$size_t$$ size,
                void (*dtor)(void*, void*),
                void *meta,
                $$size_t$$ metasize);
    \end{lstlisting}
  \end{onlyenv}

  \begin{onlyenv}<3>
    \begin{center}
      \Huge:'(\normalsize
    \end{center}
  \end{onlyenv}

  \begin{onlyenv}<4->

    \begin{lstlisting}
  struct smalloc_args {
    $$size_t$$ size;
    void (*dtor)(void*, void*);
    struct {
        void *data;
        $$size_t$$ metasize;
    } meta;
  };

  void *smalloc(struct smalloc_args *);

  #define smalloc(...)                  \
    ##smalloc(&(struct smalloc_args) {    \
    ##    __VA_ARGS__                     \
    ##})
    \end{lstlisting}

  \end{onlyenv}

\end{frame}

\begin{frame}[fragile]{Enhancing smalloc}

    \begin{lstlisting}

smart int *ptr = smalloc(sizeof (int));

smart int *ptr = smalloc(sizeof (int), my_int_dtor);

smart int *ptr = smalloc(
        sizeof (int),
        my_int_dtor,
        { my_meta, sizeof (my_meta) }
    );

smart int *ptr = smalloc(
        .size = sizeof (int),
        .dtor = my_int_dtor,
        .meta = { my_meta, sizeof (my_meta) }
    );
    \end{lstlisting}

\end{frame}

\plain{Awesome!}

\begin{frame}[fragile]{A look back on what we did}

    \verb|sfree| is the universal deallocator, just like \verb|delete| in C++

    ISO C99:
    \begin{lstlisting}
    $$whatever$$ *ptr = smalloc(
            .size = sizeof ($$whatever$$),
            .dtor = dtor_whatever
        );
    // manipulate ptr
    sfree(ptr);
    \end{lstlisting}

    C99 with GNU extensions:
    \begin{lstlisting}
    smart $$whatever$$ *ptr = smalloc(
            .size = sizeof ($$whatever$$),
            .dtor = dtor_whatever
        );
    // manipulate ptr
    \end{lstlisting}

    We could leave it at that, but...

\end{frame}

\plain{But it's still not enough.}

\begin{frame}[fragile]{Shared pointers}

    Unique pointers (no shared ownership)
    \begin{lstlisting}
  smart int *ptr = smalloc(UNIQUE, sizeof (int));
    \end{lstlisting}

    Shared pointers (shared ownership)
    \begin{lstlisting}
  void *sref(void *); // increments reference counter

  smart int *ptr = smalloc(SHARED, sizeof (int));
  smart int *ref = sref(ptr);
    \end{lstlisting}

\end{frame}

\begin{frame}[fragile]{Shared pointers: a simple example}

  \begin{lstlisting}

  smart int *ptr = smalloc(SHARED, sizeof (int));
  if (!ptr)
      abort();

  *ptr = 42;

  list_add(some_list, sref(ptr));
  list_add(some_list, sref(ptr));
  list_add(some_list, sref(ptr));

  \end{lstlisting}

\end{frame}

\begin{frame}[fragile]{Syntax sucks again}

  The boilerplate code strikes back.

  \begin{lstlisting}
  smart foo *ptr = smalloc(UNIQUE,
          sizeof (foo),
          foo_dtor);
  if (!ptr)
      abort();
  *ptr = (foo) {...};
  \end{lstlisting}

  \pause{}

  \begin{itemize}
    \item Obligatory NULL check.
    \item We almost always define a value right away.
  \end{itemize}

  Nobody wants to do that hundred of times.

\end{frame}

\begin{frame}[fragile]{Adding more syntactic sugar to the cake}

  Let's introduce two macros: \verb|unique_ptr| and \verb|shared_ptr|.

  \begin{lstlisting}
  smart $$int$$ *yes = unique_ptr($$int$$, 1);
  smart $$int$$ *yes = unique_ptr($$int$$, .value = 1);
  \end{lstlisting}

  \begin{lstlisting}
  typedef struct { float x; float y; } $$point$$;

  smart $$point$$ *zero = unique_ptr($$point$$);
  smart $$point$$ *p = shared_ptr($$point$$, { 1.3, 4.2 });
  \end{lstlisting}

\end{frame}

\begin{frame}[fragile]{Handling arrays}

  smalloc/sfree is the equivalent of new/delete in C++.

  There is no equivalent for new[] and delete[].

  We need to change the implementation of smalloc to handle arrays and change
  \verb|unique_ptr| and \verb|shared_ptr|.

  \pause{}

  \begin{lstlisting}
smart int *arr = unique_ptr(int[5], {1, 2, 3, 4, 5});
  \end{lstlisting}

  \pause{}

  \begin{lstlisting}
for ($$size_t$$ i = 0; i < array_length(arr); ++i)
    printf("%d\n", arr[i]);
  \end{lstlisting}

\end{frame}

\plain{What's next?}

\begin{frame}[fragile]{What's next?}
  \begin{itemize}
    \item weak pointers
    \item move semantics for ownership transfer
  \end{itemize}

  \begin{center}
    Any sensible contribution is welcome.

    \large \faGithub \normalsize\hspace{0.7em}
    \url{https://snai.pe/git/libcsptr}

    \large \faEnvelope \normalsize\hspace{0.7em}
    \verb|franklinmathieu+libcsptr@gmail.com|

    \verb|Snaipe| on \verb|#criterion-dev| @ \url{irc.freenode.net}

  \end{center}
\end{frame}

\plain{Questions?}

\appendix
\section{Addendum}

\subsection{Extracting the size of an array}

\begin{frame}[fragile]{Extracting the size of an array}

  Let \verb|T| be a type and \verb|dummy| be a variable of type \verb|T[1]|.

  Then:

  \begin{itemize}
    \item \verb|sizeof (dummy[0])| is the size in bytes of the compound type of T.
    \item \verb|sizeof (dummy) / sizeof (dummy[0])| is the length of the array if T is an array type, 1 otherwise.
  \end{itemize}

  Example:
  \begin{lstlisting}
  $$T$$[1] v;
  $$size_t$$ size = sizeof (v[0]);
  $$size_t$$ lenght = sizeof (v) / sizeof (v[0]);
  printf("%lu, %lu\n", size, length);
  \end{lstlisting}

  If T is an int, this prints \verb|4, 1|.

  If T is an int[20], this prints \verb|4, 20|.

  Note: If T is a multidimensional array type, the yielded length is the length of the flattened array.

\end{frame}

\end{document}
